%% How to compile:
%% $ pdflatex report.tex
%% $ bibtex report.aux
%% $ pdflatex report.tex

%% Guidelines on the ACM Latex package:
%% https://authors.acm.org/proceedings/production-information/preparing-your-article-with-latex


%% format options: manuscript, acmsmall, acmlarge, acmtog, sigconf, siggraph, sigplan, sigchi
%% used formats: manuscript, sigplan
\documentclass[
  format=manuscript,
  screen=true,
  review=false,
  nonacm=true,
  timestamp=true,
  balance=false]{acmart}
\setcopyright{rightsretained}

\author{Alexander Christensen}
\title{C Scoped Enum Enhancements}
\email{alex\_c007@hotmail.com}

%% \citestyle{acmauthoryear}


%% ===================== %%
%% === CODE LISTINGS === %%
%% ===================== %%
\usepackage{listings}
\usepackage{color}

\definecolor{backgroundblue}{rgb}{0.93, 0.93, 1.0}
\definecolor{mygreen}{rgb}{0,0.6,0}
\definecolor{mygray}{rgb}{0.5,0.5,0.5}
\definecolor{mymauve}{rgb}{0.58,0,0.82}

\definecolor{LightCyan}{rgb}{0.88,1,1}
\definecolor{bluekeywords}{rgb}{0.13,0.13,1}
\definecolor{greencomments}{rgb}{0,0.5,0}
\definecolor{turqusnumbers}{rgb}{0.17,0.57,0.69}
\definecolor{redstrings}{rgb}{0.5,0,0}
\definecolor{red}{rgb}{0.5,0.0,0.0}
\definecolor{blue}{rgb}{0.0,0.5,1.0}
\definecolor{green}{rgb}{0.0,0.5,0.0}


% basic settings, can be overrided
\lstset{
  basicstyle=\ttfamily\small,
  breaklines=true,
  columns=fullflexible,
  escapeinside={\%},
  frame=none,
  backgroundcolor=\color{backgroundblue},
  showspaces=false,
  keepspaces=true,
  showstringspaces=false,
  showtabs=false,
  numbers=left,
  aboveskip=-3pt,
  sensitive=true
}


% nice horizontal line "-------- <text> --------", used above a listing
\def\headline#1{\begin{minipage}{36em}\vspace{4mm}\hrulefill\quad\lower.3em\hbox{#1}\quad\hrulefill\end{minipage}}

% horizontal line, used below a listing
\newcommand{\lstline}{\vspace{-3mm}\hrulefill\vspace{2mm}\newline}


% fancy insert listing command
% example of usage: `\customlisting{FSharp}{Some Function}{<file.fsx>}`
\newcommand{\customlisting}[3]{\lstinputlisting[language=#1,name=#2]{#3}\lstline}


% Insert the code, e.g. from a file like this:
% \lstinputlisting[language=<language>]{<input_file>}
% \lstinputlisting[language=<language>, firstline=X, lastline=Y]{<input_file>}

% Or directly write the code in the .tex document:
% \begin{lstlisting}[language=<language>]

\lstdefinelanguage{Cpp}{
  morekeywords={float, int, double, uint, bool, if, for, else, void, class,
                struct, private, protected, public, enum, const, char, case,
                default, return, switch},
  keywordstyle=\color{bluekeywords},
  morecomment=[l][\color{greencomments}]{///},
  morecomment=[l][\color{greencomments}]{//},
  morecomment=[l][\color{redstrings}]{\#},
  morecomment=[s][\color{greencomments}]{{/*}{*/}},
  morestring=[b]",
  stringstyle=\color{redstrings},
  %title={\headline{\small C++ - \textit{\lstname}}}
}


%% =========================== %%
%% === ADDITIONAL PACKAGES === %%
%% =========================== %%
% Load with some options, i.e. \usepackage[colorlinks=true,linkcolor=blue]{hyperref} or blank
% \usepackage{hyperref}
%\usepackage[colorlinks=true,linkcolor=blue, linktocpage]{hyperref}


%% ================ %%
%% === DOCUMENT === %%
%% ================ %%
\begin{document}

%% \abstract{
%% This is a very abstract abstract.
%% }

\maketitle
\tableofcontents

\section{Introduction}

\section{Motivation and Scope}

The initial motivation for this proposal was the lack of a good way in the standard
library to convert an enum value to a proper string representation. Very often, I
have found a need to log an enum for various purposes, and every time I create a new
enum type I have to write such a function again. An example:

\begin{lstlisting}[language=Cpp]
enum class GraphicsApiType { none, opengl, vulkan };

const char* get_api_type_string(GraphicsApiType apiType) {
    switch (apiType) {
    case GraphicsApiType::none:   return "none";
    case GraphicsApiType::opengl: return "opengl";
    case GraphicsApiType::vulkan: return "vulkan";
    default: return "<unrecognized>";
    }
}
\end{lstlisting}

\noindent
This is cumbersome to maintain, for every time a value is added or modified inside
the enum type, this other function has to be modified as well. But besides this,
a potential \textit{run-time} error may be introduced when an invalid enum is
provided.

A session of browsing around sites such as StackOverflow revealed sometimes quite
vividly imaginary answers for how to circumvent this limitation in the language:

\begin{enumerate}
\item \url{https://stackoverflow.com/questions/11421432/how-can-i-output-the-value-of-an-enum-class-in-c11}
\item \url{https://stackoverflow.com/questions/1390703/enumerate-over-an-enum-in-c}
\item \url{https://stackoverflow.com/questions/6281461/enum-to-string-c}
\item \url{https://stackoverflow.com/questions/201593/is-there-a-simple-way-to-convert-c-enum-to-string}
\end{enumerate}


\section{Impact on the Standard}

\section{Design Decisions}

\section{Technical Specifications}

\section{Acknowledgements}

\section{References}




%% \bibliographystyle{ACM-Reference-Format}
%% \bibliography{references}

\end{document}

