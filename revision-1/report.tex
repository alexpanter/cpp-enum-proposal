%% How to compile:
%% $ pdflatex report.tex
%% $ bibtex report.aux
%% $ pdflatex report.tex

%% Guidelines on the ACM Latex package:
%% https://authors.acm.org/proceedings/production-information/preparing-your-article-with-latex


%% format options: manuscript, acmsmall, acmlarge, acmtog, sigconf, siggraph, sigplan, sigchi
%% used formats: manuscript, sigplan
\documentclass[
  format=manuscript,
  screen=true,
  review=false,
  nonacm=true,
  timestamp=true,
  balance=false]{acmart}
\setcopyright{rightsretained}

\author{Alexander Christensen}
\title{C++ Scoped Enum Enhancements}
\email{<alex\_c007@hotmail.com>}

%% \citestyle{acmauthoryear}


%% ===================== %%
%% === CODE LISTINGS === %%
%% ===================== %%
\usepackage{listings}
\usepackage{color}

\definecolor{backgroundblue}{rgb}{0.93, 0.93, 1.0}
\definecolor{mygreen}{rgb}{0,0.6,0}
\definecolor{mygray}{rgb}{0.5,0.5,0.5}
\definecolor{mymauve}{rgb}{0.58,0,0.82}

\definecolor{LightCyan}{rgb}{0.88,1,1}
\definecolor{bluekeywords}{rgb}{0.13,0.13,1}
\definecolor{greencomments}{rgb}{0,0.5,0}
\definecolor{turqusnumbers}{rgb}{0.17,0.57,0.69}
\definecolor{redstrings}{rgb}{0.5,0,0}
\definecolor{red}{rgb}{0.5,0.0,0.0}
\definecolor{blue}{rgb}{0.0,0.5,1.0}
\definecolor{green}{rgb}{0.0,0.5,0.0}


% basic settings, can be overrided
\lstset{
  basicstyle=\ttfamily\small,
  breaklines=true,
  columns=fullflexible,
  escapeinside={\%},
  frame=none,
  backgroundcolor=\color{backgroundblue},
  showspaces=false,
  keepspaces=true,
  showstringspaces=false,
  showtabs=false,
  numbers=left,
  aboveskip=-3pt,
  sensitive=true
}


% nice horizontal line "-------- <text> --------", used above a listing
\def\headline#1{\begin{minipage}{36em}\vspace{4mm}\hrulefill\quad\lower.3em\hbox{#1}\quad\hrulefill\end{minipage}}

% horizontal line, used below a listing
\newcommand{\lstline}{\vspace{-3mm}\hrulefill\vspace{2mm}\newline}


% fancy insert listing command
% example of usage: `\customlisting{FSharp}{Some Function}{<file.fsx>}`
\newcommand{\customlisting}[3]{\lstinputlisting[language=#1,name=#2]{#3}\lstline}


% Insert the code, e.g. from a file like this:
% \lstinputlisting[language=<language>]{<input_file>}
% \lstinputlisting[language=<language>, firstline=X, lastline=Y]{<input_file>}

% Or directly write the code in the .tex document:
% \begin{lstlisting}[language=<language>]

\lstdefinelanguage{Cpp}{
  morekeywords={float, int, double, uint, bool, if, for, else, void, class,
                struct, private, protected, public, enum, const, char, case,
                default, return, switch, template, constexpr, noexcept, auto,
                typename},
  keywordstyle=\color{bluekeywords},
  morecomment=[l][\color{greencomments}]{///},
  morecomment=[l][\color{greencomments}]{//},
  morecomment=[l][\color{redstrings}]{\#},
  morecomment=[s][\color{greencomments}]{{/*}{*/}},
  morestring=[b]",
  stringstyle=\color{redstrings},
  %title={\headline{\small C++ - \textit{\lstname}}}
}


%% =========================== %%
%% === ADDITIONAL PACKAGES === %%
%% =========================== %%
% Load with some options, i.e. \usepackage[colorlinks=true,linkcolor=blue]{hyperref} or blank
% \usepackage{hyperref}
%\usepackage[colorlinks=true,linkcolor=blue, linktocpage]{hyperref}


%% ================ %%
%% === DOCUMENT === %%
%% ================ %%
\begin{document}

%% \abstract{
%% This is a very abstract abstract.
%% }

\maketitle
\tableofcontents

\section{Introduction}

This proposal is the first revision of "C++ Scoped Enum Proposal", which was circulated
in the C++ proposals email list during the winter of 2021/2022. The goal of writing
this proposal is to gain feedback and insights, which may then be used for further
revisions.

\noindent
Note: This proposal targets only \underline{scoped enums}, and does not provide any
effort to address \textit{unscoped} enums.

\vspace{2mm}\noindent
Given my limited experience (actually: none) in writing C++ proposals, I know what I
\underline{would like} the proposal to address and introduce to the core language,
but not necessarily \underline{how it should be done}. The suggested notations
provided are, thus, merely suggestions - and not concrete proposals for a concious
choice of syntax.


\section{Motivation and Scope}

The initial motivation for this proposal was the lack of a good way in the standard
library to convert an enum value to a proper string representation. Very often, I
have found a need to log an enum for various purposes, and every time I create a new
enum type I have to write such a function again. An example:\vspace{2mm}

\begin{lstlisting}[language=Cpp]
enum class GraphicsApiType { none, opengl, vulkan };

const char* get_api_type_string(GraphicsApiType apiType) {
    switch (apiType) {
    case GraphicsApiType::none:   return "none";
    case GraphicsApiType::opengl: return "opengl";
    case GraphicsApiType::vulkan: return "vulkan";
    default: return "<unrecognized>";
    }
}
\end{lstlisting}

\noindent
This is cumbersome to maintain, for every time a value is added or modified inside
the enum type, this other function has to be modified as well. But besides this,
a potential \textit{run-time} error may be introduced when an invalid enum is
provided.

A session of browsing around sites such as StackOverflow revealed sometimes quite
vividly imaginary answers for how to circumvent this limitation in the language:

\begin{enumerate}
\item \url{https://stackoverflow.com/questions/11421432/how-can-i-output-the-value-of-an-enum-class-in-c11}
\item \url{https://stackoverflow.com/questions/1390703/enumerate-over-an-enum-in-c}
\item \url{https://stackoverflow.com/questions/6281461/enum-to-string-c}
\item \url{https://stackoverflow.com/questions/201593/is-there-a-simple-way-to-convert-c-enum-to-string}
\end{enumerate}

\noindent
Quite intuitively, two key observations were made:

\begin{itemize}
\item This proposal introduces \underline{changes to the core language}.
\item While fixing enum to string functionality, there's a lot more that we could do at the same time.
\end{itemize}

\subsection{Scope for this proposal}

Only \textit{scoped enums} are extended with this proposal, because they are fully
type safe, meaning that we can theoretically perform all necessary computations needed
for this proposal \textit{compile-time}, with no risk of throwing exceptions or using
any other runtime mechanism.


\section{Impact on the Standard}

This proposal requires new language features, because certain enum extensions cannot
be covered by simply writing a library. The impacts are listed below with a subsection
for each of them.



\section{Convert scoped enum to string}

Two ways of obtaining a string is suggested - the simple one, and a fully-qualified
one which contains namespaces, classes, and other scopes:\vspace{2mm}

\begin{lstlisting}[language=Cpp]
namespace graphics {
    enum class GraphicsApiType { none, opengl, vulkan };
}
std::cout << std::enum_str<GraphicsApiType::opengl>() << std::endl;
// --> "opengl"
std::cout << std::enum_str_full<GraphicsApiType::opengl>() << std::endl;
// "graphics::GraphicsApiType::opengl"
\end{lstlisting}

\noindent
The return type itself gives us a number of possibilites, but two are probably worthy
of consideration: \texttt{const char*} and \texttt{std::string\_view}.

\section{Concept for "is-an-enum"}

We can with C++-23 create a concept which is useful for template substitution:\vspace{2mm}

\begin{lstlisting}[language=Cpp]
template<typename T>
concept scoped_enum = std::is_scoped_enum_v<T>;

template<std::scoped_enum T>
class foo { int bar; };

// Try it out:
foo<GraphicsApiType> f1; // <-- success!
foo<unsigned long> f2; // <-- fails to compile!
\end{lstlisting}


\section{Obtain the underlying type}

In current C++, when needing to compare an enum value to its underlying type,
a relatively verbose expression is needed which involves a "\texttt{static\_cast}"
and possible also an "\texttt{std::underlying\_type}". This has been somewhat
mitigated by the introduction of \texttt{std::to\_underlying}, available with C++23,
but this proposal adds a new template meta-function which does almost the same
thing:\vspace{2mm}

\begin{lstlisting}[language=Cpp]
template<scoped_enum E>
struct enum_type {
    using type = std::underlying_type<E>::type;
};

template<scoped_enum E>
using enum_type_t = std::enum_type<E>::type;

// Try it out:
enum class Color : int { red, green, blue, yellow };
std::enum_type_t<Color> yellow = Color::yellow;
\end{lstlisting}

The benefit of this meta-function, as opposed to \texttt{std::to\_underlying}, is
that this meta-function does not work for unscoped enumerations. This is because
\texttt{std::to\_underlying} uses \texttt{std::underlying\_type}, while this new
meta-function uses the newly proposed concept \texttt{std::scoped\_enum}.
This creates a more accentuated difference between scoped and unscoped enums, and
thereby encourages the use of scoped enums for enhanced type safety and scope
safety.


\section{Get the underlying value}

Similar to how getting an underlying type can be tedious, this proposal also
addresses the need for obtaining the underlying value of a scoped enum value:\vspace{2mm}

\begin{lstlisting}[language=Cpp]
template<scoped_enum Enum>
constexpr std::enum_type_t<Enum> enum_value(Enum e) noexcept
{
    // TODO: Error - use enum_cast instead!
    return static_cast<std::enum_type_t<Enum>>(e);
}
\end{lstlisting}


\section{Safe enum conversion (language feature)}

Sometimes, a conversion from an underlying type to an enum value may be approved
without being valid. And there exists no compile-time feature to check for this.
An example of how it may go wrong:\vspace{2mm}

\begin{lstlisting}[language=Cpp]
enum class MyScopedEnum : int
{
    value_0 = 0, value_1 = 1, value_2 = 2, value_3 = 3, value_43 = 43, value_57 = 57
};

constexpr MyScopedEnum test = static_cast<MyScopedEnum>(56);
\end{lstlisting}

\noindent
Even if we explicitly denote our variable \texttt{constexpr}, the compiler is not
able to enforce correct behaviour, and so we obtain undefined behaviour. The topic
is discussed on StackOverflow:

\noindent
\url{https://stackoverflow.com/questions/17811179/safe-way-to-convert-an-integer-in-an-enum}

\vspace{3mm}\noindent
This proposal thus defines a template meta-function which fails at
\underline{compile-time} if the conversion is invalid. It might be possible to also
define a \textit{run-time} mechanism which throws an exception, but this revision
does not discuss that topic further. Proposed syntax:\vspace{2mm}

\begin{lstlisting}[language=Cpp]
template<std::scoped_enum T>
constexpr bool is_enum_convertible(std::enum_type_t<T> t) {
    return ...; // <-- compiler implements this
};

template<std::scoped_enum T>
consteval T enum_cast(std::enum_type_t<T> t) {
    static_assert(std::is_enum_convertible<T>(t));
    return static_cast<T>(t);
}
\end{lstlisting}






\section{Iterating through a scoped enum}

Another interesting scenario is if an enum contains "identifiers", where for each we
want to create some data structure and add that to some general storage. An example
of such storage is shown below, using the \texttt{scoped\_enum} concept proposed
earlier:\vspace{2mm}

\begin{lstlisting}[language=Cpp]
template<std::scoped_enum E, typename T>
class EnumMap {
public:
    bool has_value(E e) { return mEnumMap.contains(e); }
    void add_value(E e, T&& t) { mEnumMap[e] = t; }
    std::optional<T> get_value(E e) {
        if (auto it = mEnumMap.find(e); it != mEnumMap.end()) {
            return { it->second };
        } else { return std::nullopt; }
    }
private:
    std::map<E, T> mEnumMap;
};
\end{lstlisting}

We can then obtain a "bi-directional" iterator, which allows us to iterate through
all values in the enumeration and add them to the storage:\vspace{2mm}

\begin{lstlisting}[language=Cpp]
void fill(EnumMap<auto T, auto V>& em) {
    for (T t : std::enum_values<T>)
        em.add value(t, V{});
}
\end{lstlisting}


\section{Enum Cardinality}

Another feature which might sometimes be useful, is the ability to count the number
of values inside an enumeration. This proposal, for the sake of \textit{completeness},
defines a template meta-function which does this:\vspace{2mm}

\begin{lstlisting}[language=Cpp]
template<scoped_enum E>
struct enum_cardinality
{
    static constexpr std::size_t value = 1; // <-- compiler implements this
};

template<scoped_enum E>
std::size_t enum_cardinality_v = std::enum_cardinality<E>::value;

// Try it out:
enum class Weekday { monday, tuesday, /* etc. */ };
static_assert(std::enum_cardinality_v<Weekday> == 7);
\end{lstlisting}


\section{Enum position}

We might also want to know the position or the \textit{index} of an enum value
inside an enumeration. As with \texttt{std::enum\_cardinality}, we need to rely
on the compiler to provide an implementation, as well as a requirement of strict
compile-time evaluation:\vspace{2mm}

\begin{lstlisting}[language=Cpp]
template<std::scoped_enum T, T E>
struct enum_position {
    // TODO: Perhaps we _can_ actually use some template meta-programming to do this?
    static constexpr std::size_t value = 0; // <-- compiler implements this
};

template<std::scoped_enum T, T E>
std::size_t enum_position_v = std::enum_position<T, E>::value;

// Try it out:
enum class Fruit { banana = 1, apple = 2, orange = 4, clementine = 8 };
static assert(std::enum_value_v<Fruit::orange> == 4);
static assert(std::enum_position<Fruit, Fruit::orange> == 2);
\end{lstlisting}

\noindent
With both \texttt{enum\_cardinality} and \texttt{enum\_position} implemented, we
can create e.g. create a priority mapper with a priority for each enum value:\vspace{2mm}

\begin{lstlisting}[language=Cpp]
template<std::scoped_enum T>
class EnumPriorityMapper {
public:
    unsigned int GetPriority(T t) {
        return mPriorities[std::enum_position_v<T, t>];
    }
    void SetPriority(T t, unsigned int p) {
        mPriorities[std::enum_position_v<T, t>] = p;
    }
private:
    unsigned int mPriorities[std::enum_cardinality_v<T>];
};
\end{lstlisting}


\section{Weak ordering}

With a good comparison operator in place, it becomes straight-forward to test
the order in which values appear inside an enumeration. Using the "spaceship operator"
introduced with C++20, this proposal defines two three-way comparison functions:\vspace{2mm}

\begin{lstlisting}[language=Cpp]
namespace std {
    // ...
}

template<std::scoped_enum T>
constexpr std::weak_ordering operator<=>(T t, std::integral auto i) {
    return std::enum_value(t) <=> i;
}

template<std::scoped_enum T>
std::weak_ordering operator<=>(T s, T t) {
    return std::enum_value(s) <=> std::enum_value(t);
}

// Try it out:
enum class MyScopedEnum : int
{
    value_0 = 0, value_1 = 1, value_2 = 2, value_3 = 3, value_43 = 43, value_57 = 57
};
static_assert((MyScopedEnum::value_0 <=> 0) == 0);
static_assert((1 <=> MyScopedEnum::value_2) < 0);
static_assert((MyScopedEnum::value_2 <=> 1) > 0);
\end{lstlisting}








\section{Proposed Notation Summary}




\section{Design Decisions}



\section{Technical Specifications}

\section{Acknowledgements}

\section{References}




%% \bibliographystyle{ACM-Reference-Format}
%% \bibliography{references}

\end{document}

