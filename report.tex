\documentclass[a4paper, 12pt]{article}

\usepackage[margin={1.1in, 1.0in}]{geometry}
\usepackage{layout}

\usepackage{alltt}
\usepackage[utf8]{inputenc}
\usepackage{hyperref}

\title{C++ Scoped Enum proposal}
\author{\small Alexander Christensen}
\date{\vspace{-2mm}\small \today}

\begin{document}
\maketitle

Following discussions on StackOverflow, and the interest generated, it seems
that there could be motivation for extending the scoped enums to allow for
more flexible operations. Such questions, and their often vividly imaginary
answers, are found here:

\begin{enumerate}
\item \href{https://stackoverflow.com/questions/11421432/how-can-i-output-the-value-of-an-enum-class-in-c11}{how-can-i-output-the-value-of-an-enum-class-in-c11}
\item \href{https://stackoverflow.com/questions/1390703/enumerate-over-an-enum-in-c}{enumerate-over-an-enum-in-c}
\item \href{https://stackoverflow.com/questions/6281461/enum-to-string-c}{enum-to-string-c}
\item \href{https://stackoverflow.com/questions/201593/is-there-a-simple-way-to-convert-c-enum-to-string}{is-there-a-simple-way-to-convert-c-enum-to-string}
\end{enumerate}

Following the apparent interest in the subject, this proposal aims to bring
into the C++ ($>$= 20) standard library improved \textit{compile-time} support
for scoped enums (as signified by "\texttt{enum class}"), in particular
enumeration, and cast to string or underlying type. As such, \texttt{type\_traits}
or other RTTI-techniques will not be considered.

For maintaining backwards-compatibility with C, this proposal \textit{only}
targets the scoped enums, and \textbf{not} the C-style (unscoped) enums.


\section{Enumeration}

Sometimes it may be desirable to enumerate through a scoped enum to know its
values. One particular use case is when an enum is used as a template parameter:

\begin{alltt}\footnotesize
template<"enum" T, typename V>
requires std::is\_default\_constructible\_v<V>
class foo \{
public:
  bool has\_value(T t) \{ return mEnumMap.contains(t); \}
  void add\_value(T t, V&& v) \{ mEnumMap[t] = v; \}
private:
  std::map<T, V> mEnumMap;
\};

void fill(foo<auto T, auto V>& f) \{
  auto begin = std::enum\_iterator<T>.cbegin();
  auto end = std::enum\_iterator<T>.cend();
  for (auto it = begin; it != end; it++) { f.add\_value(*it, V\{\}); }

  // Alternatively:
  for (auto t : std::enum\_values\_v<T>) { f.add\_value(t, V\{\}); }
\}
\end{alltt}

In such a system, we may maintain a dictionary of various enum members and bind
them to a particular type \texttt{"V"}.

\section{Underlying type}

In current C++ standard, when needing to compare an enum value with its
underlying type, a relatively verbose code expression is required:

\begin{alltt}\footnotesize
// Check if enum value is 0
auto my\_enum\_value = MyEnum::Value;
bool is\_zero = static\_cast<std::underlying\_type\_t<MyEnum>>(my\_enum\_value) == 0;
\end{alltt}

\noindent
This document proposes a simpler syntax. Assume we have a definition
"\texttt{enum class MyEnum}":

\begin{alltt}\footnotesize
// Check if enum value is 0
auto my\_enum\_value = MyEnum::Value;
bool is\_zero = std::enum\_value<my\_enum\_value> == 0;
\end{alltt}

\noindent
Seemingly, this does not yield much benefit since we just removed a
\texttt{static\_cast} and replaced one template meta-function with another.
But without the \texttt{static\_cast} we can have the compiler auto-generate
an implicit operator overload, similar to "\texttt{ToInt()}" found in some
object-oriented languages. And with that in mind, we should be able to write
our expression above in much simpler terms:

\begin{alltt}\footnotesize
// Check if enum value is 0
auto my\_enum\_value = MyEnum::Value;
bool is\_zero = (my\_enum\_value == 0);
\end{alltt}

\noindent
Sometimes, we may also wish to convert the other way around. But for this
we need to make sure that our input value does actually represent a valid
enum value. Additionally, this document proposes the following syntax:

\begin{alltt}\footnotesize
enum class Color : int \{ red = 0, green, blue \};

optional<Color> newColor = enum\_cast<Color>(3);
static\_assert(newColor == std::nullopt);
\end{alltt}



\section{Scoped enum cardinality}

In some usage scenarios it can be useful to retrieve the number of values
declared inside a scoped enum. Current possibilities includes only hacks,
so the following syntax is proposed:

\begin{alltt}\footnotesize
enum class Weekdays \{ monday, tuesday /*, etc. */ \};

// NOTE: Possible alternative namings - "enum\_count\_v" or "enum\_size\_v".
static\_assert(std::enum\_cardinality\_v<Weekdays> == 7);
\end{alltt}



\section{String representation}

Sometimes, for debugging purposes, we might want to be able to output to a
terminal the string name of an enum value. Instead of creating such a function
ourselves and maintaining it when new values are added to the enum or renamed,
better yet to let the compiler do the heavy-lifting for us. Proposed syntax:

\begin{alltt}\footnotesize
enum class Color \{ red, green, blue \};

// Two alternatives:
std::cout << std::enum\_string<Color::red>() << std::endl;
std::cout << std::enum\_string\_v<Color::red> << std::endl;
\end{alltt}


\subsection{Runtime deduction (Superfluous chapter??)}

The example above assumes we know which enum value we want to print at
compile-time, but this is something we might not always know. So the following
strategy is proposed:

\begin{enumerate}
\item We use \texttt{std::enum\_string<T>(T t)} with some value, e.g. as input
  to a function.
\item The compiler detects this use and generates a lookup table (ie.
  \texttt{const char**}) inside the readonly section of our target binary.
\item At runtime, with little overhead, we make a lookup into this table and
  retrieves our string.
\end{enumerate}

\noindent
Creating such a table requires that each enum value has a certain
\textit{position} within its containing scoped enum, which we may query.
This would necessitate the following interface:

\begin{alltt}\footnotesize
namespace std \{
  template<enum\_type T> size\_t enum\_position(T t);
  template<enum\_type> const char* enum\_indexed\_string(integral auto i);

  template<>
  struct enum\_string<enum\_type T> \{
    const char* operator()(T t);
  \};
\}
\end{alltt}

\noindent
The following code examplifies the usage:

\begin{alltt}\footnotesize
void printColor(Color c) \{
  std::cout << std::enum\_string<Color>(c) << std::endl;
\}

// Alternatively, if "std::cout" is provided with a meta-function overload:
void printColor(Color c) \{ std::cout << c << std::endl; \}

// Testing:
printColor(Color::red); // Should output "<namespace::>Color::red"
\end{alltt}



\newpage
\section{Overview}

This is the complete list of proposed functions, types, and meta-functions:

\begin{alltt}\footnotesize
// header "<experimental/scoped\_enum>" -- or -- "<experimental/enum>"

// Alternatively:
export module std.scoped\_enum; // -- or --
export module std.enum;

namespace std \{

  // Evaluates whether T is the name of a declared scoped enum type
  template<typename T>
  concept enum\_type; // = /* implementation uncertain */

  // Checks if "V" is the underlying type for a declared scoped enum "T".
  template<enum\_type T, is\_integral\_v V>
  concept enum\_has\_type = std::is\_same\_v<std::underlying\_type\_t<T>, V>;

  // A forward-iterator of all values inside a declared scoped enum
  template<enum\_type T>
  struct enum\_iterator;

  // A container with all values inside a declared scoped enum,
  // with methods `begin()`, `end()`, `cbegin()`, and `cend()` for
  // supporting forward iterating through the values by using the
  // "enum\_iterator" defined above.
  template<enum\_type T>
  struct enum\_values;

  // Retrieves the underlying value represented by an enum value,
  // similar to "static\_cast<underlying\_type\_t<T>".
  template<enum\_type T, is\_integral\_v V>
  V enum\_value(T t);

  template<enum\_type T>
  bool enum\_try\_cast(integral auto v); // = /* implementation uncertain */

  // Retrieve the cardinality (ie. number of values) inside a declared
  // scoped enum.
  template<enum\_type T>
  struct std::enum\_cardinality \{ constexpr size\_t value; \};

  template<enum\_type T>
  size\_t std::enum\_cardinality\_v;

  // Retrieve a fully-qualified string representation of the provided
  // scoped enum value.
  template<enum\_type T>
  const char* enum\_string(T t); // = /* implementation compiler-specific */

  // Compare a scoped enum value with its underlying type (spaceship-operator)
  template<enum\_type T, is\_integral\_v V>
  weak\_ordering operator<=>(T lhs, V v);

  // Stream overload
  template<enum\_type T>
  ostream& operator<<(ostream& stream, T t) \{
    return stream << enum\_string<T>(t);
  \}

\} // namespace std

// Typesafe cast from underlying value to declared scoped enum value,
// including a check for whether the provided value is valid.
template<std::enum\_type T>
std::optional<T> enum\_cast(std::integral auto v) \{
  T out;
  if (std::enum\_try\_cast<T>(&out, v)) return out;
  else return std::nullopt;
\}

\end{alltt}




\end{document}
