\documentclass[a4paper, 12pt]{article}

\usepackage[margin={1.1in, 1.0in}]{geometry}
\usepackage{layout}
\usepackage[bottom]{footmisc}
\usepackage{fancyhdr}
\usepackage{lastpage}

\usepackage{alltt}
\usepackage[utf8]{inputenc}
\usepackage{hyperref}
\usepackage{nameref}

\makeatletter
\newcommand*{\currentname}{\@currentlabelname}
\makeatother

\newcommand{\mytitle}{C++ Scoped Enum proposal}
\title{\mytitle}
\author{\small Alexander Christensen}
\date{\vspace{-2mm}\small \today}

\pagestyle{fancy}
%\fancyhf{}
\lhead{\currentname}
\rhead{\mytitle}
\cfoot{\footnotesize Page \thepage \hspace{1pt} of \pageref{LastPage}}

\fancypagestyle{firststyle}
{
   \fancyhf{}
   \cfoot{\footnotesize Page \thepage\ of \pageref{LastPage}}
   \renewcommand{\headrulewidth}{0pt}
}

\begin{document}
\maketitle
\thispagestyle{firststyle}

Following discussions on StackOverflow, and the interest generated, it seems
that there could be motivation for extending the scoped enums to allow for
more flexible operations. Such questions, and their often vividly imaginary
answers, are found here:

\begin{enumerate}
\item \href{https://stackoverflow.com/questions/11421432/how-can-i-output-the-value-of-an-enum-class-in-c11}{how-can-i-output-the-value-of-an-enum-class-in-c11}
\item \href{https://stackoverflow.com/questions/1390703/enumerate-over-an-enum-in-c}{enumerate-over-an-enum-in-c}
\item \href{https://stackoverflow.com/questions/6281461/enum-to-string-c}{enum-to-string-c}
\item \href{https://stackoverflow.com/questions/201593/is-there-a-simple-way-to-convert-c-enum-to-string}{is-there-a-simple-way-to-convert-c-enum-to-string}
\end{enumerate}

Following the apparent interest in the subject, this proposal aims to bring
into the C++ ($>$= 20) standard library improved \textit{compile-time} support
for scoped enums (as signified by "\texttt{enum class}"), in particular
enumeration, and cast to string or underlying type. As such, the
\texttt{typeinfo}-header or other RTTI-techniques will not be considered.
For maintaining backwards-compatibility with C, this proposal \textit{only}
targets the scoped enums, and \textbf{not} the C-style (unscoped) enums.


\section{Substitution failure}

Since the introduction of SFINAE, we have been empowered with manual
template-substitution control. Because this proposal targets the modern C++
approach however, this proposal defines a concept instead. Similar to
\texttt{typename} or \texttt{class}, we wish to perform a substitution when
we know that a provided template-type is a scoped enum:

\begin{alltt}\footnotesize
template<typename T>
concept std::enum\_type = /* static check to verify if T is a scoped enum */

// Alternatively defined (shorter version) as
template<typename T>
concept std::enum;
\end{alltt}

\noindent
We may then use this concept to constrain template-substitution:

\begin{alltt}\footnotesize
template<std::enum T>
class MyClass \{ /* ... */ \};
\end{alltt}

\subsection{C++-23}

Since C++23 adds a meta-function \texttt{std::is\_scoped\_enum}, it is
evident that the language is already developing in the direction of this
proposal. And as such, we would be able to trivially define our concept as
such:

\begin{alltt}\footnotesize
template<typename T>
concept std::enum = std::is\_scoped\_enum\_v<T>;
\end{alltt}

\noindent
Whether this addition to the language would be useful is up for discussion.
Arguably, having a concept could potentially provide better compiler-output
on substitution failure.



\section{Enumeration}

Sometimes it may be desirable to enumerate through a scoped enum to know its
values. One particular use case is when an enum is used as a template parameter:

\begin{alltt}\footnotesize
template<std::enum T, typename V>
requires std::is\_default\_constructible\_v<V>
class foo \{
public:
  bool has\_value(T t) \{ return mEnumMap.contains(t); \}
  void add\_value(T t, V&& v) \{ mEnumMap[t] = v; \}
private:
  std::map<T, V> mEnumMap;
\};

void fill(foo<auto T, auto V>& f) \{
  auto begin = std::enum\_iterator<T>.cbegin();
  auto end = std::enum\_iterator<T>.cend();
  for (auto it = begin; it != end; it++) { f.add\_value(*it, V\{\}); }

  // Alternatively:
  for (auto t : std::enum\_values\_v<T>) { f.add\_value(t, V\{\}); }
\}
\end{alltt}

In such a system, we can maintain a dictionary of various enum members and bind
them to a particular type \texttt{"V"}.

\section{Underlying type}

In current C++ standard, when needing to compare an enum value with its
underlying type, a relatively verbose code expression is required:

\begin{alltt}\footnotesize
// Check if enum value is 0
auto my\_enum\_value = MyEnum::Value;
bool is\_zero = static\_cast<std::underlying\_type\_t<MyEnum>>(my\_enum\_value) == 0;
\end{alltt}

\noindent
This document proposes a simpler syntax. Assume we have a definition
"\texttt{enum class MyEnum}":

\begin{alltt}\footnotesize
// Check if enum value is 0
auto my\_enum\_value = MyEnum::Value;
bool is\_zero = std::enum\_value<my\_enum\_value> == 0;
\end{alltt}

\noindent
This is definitely better than the first example since we have replaced
the\\ \texttt{static\_cast<std::underlying\_type>} with a simpler interface.
But we can do even better, especially with C++-20, if we provide an overload
to the "spaceship"-operator (three-way comparison), for comparing an enum value
with its underlying type:

\begin{alltt}\footnotesize
template<std::enum T>
std::weak\_ordering operator<=>(T t, std::integral auto i);

// Check if enum value is 0
auto my\_enum\_value = MyEnum::Value;
bool is\_zero = (my\_enum\_value == 0);
\end{alltt}

\noindent
Sometimes, we may also wish to convert the other way around. But for this
we need to make sure that our input value can actually be used as a valid
conversion. Thus, this document proposes an exception-safe \texttt{enum\_cast},
similar to \texttt{dynamic\_cast} but with the added benefit of being strictly
\textit{compile-time} evaluated:

\begin{alltt}\footnotesize
enum class Color : int \{ red = 0, green, blue \};

optional<Color> newColor = enum\_cast<Color>(3);
static\_assert(newColor == std::nullopt);
\end{alltt}



\section{Scoped enum cardinality}

In some usage scenarios it can be useful to retrieve the number of values
declared inside a scoped enum. Current possibilities includes only hacks,
so the following syntax is proposed:

\begin{alltt}\footnotesize
enum class Weekday \{ monday, tuesday /*, etc. */ \};

// NOTE: Possible alternative namings - "enum\_count\_v" or "enum\_size\_v".
static\_assert(std::enum\_cardinality\_v<Weekday> == 7);
\end{alltt}



\section{String representation}

Sometimes, for debugging purposes, we might want to be able to output to a
terminal the string name of an enum value, possibly \textit{fully-qualified}
\footnote{Including all containing namespaces, classes, structs, or modules.}.
Instead of creating such a function ourselves and maintaining it when new
values are added to the enum or renamed, better yet to let the compiler do
the heavy-lifting for us. Proposed syntax:

\begin{alltt}\footnotesize
enum class Color \{ red, green, blue \};

void printColor(Color c) \{
  std::cout << std::enum\_string\_v<c> << std::endl;
\}

printColor(Color::red); // Should output "Color::red"
\end{alltt}


\subsection{Failed runtime deduction}

If converting to scoped enum from an invalid underlying type, a string with
the text \texttt{"<null>"} could be inserted. This would only happen if the
conversion was not performed through the safe \texttt{enum\_cast} but through
an \textit{unsafe} C-style cast (or an unchecked \texttt{static\_cast}).\\

\noindent
Let \texttt{"\_a"} represent an integral to enum conversion which
\textit{might not} be valid, such as \texttt{"(Color)(rand())"}. The following
piece of code \underline{should} yield a compiler error, through either
detectible non-\texttt{consteval} behaviour -or- compile-time detectible
invalid conversion:

\begin{alltt}\footnotesize
// This should fail compilation:
const char* str = std::enum\_string<Color>(\_a);
\end{alltt}

\noindent
This would build upon the \texttt{enum\_cast} returning a valid
\texttt{optional<T>} or through constant propagation by the compiler. In
either case, \texttt{enum\_string\_v} should be strictly \textit{compile-time}
evaluated. Suggested implementation is to generate a conversion table
(\texttt{"const char**"}) inside the readonly section of the target binary.
\footnote{TODO: Would this be a good idea?}


\section{Enum position}

In addition to a scoped enum's cardinality, this proposal also presents a
function to get the \textit{position} of an individual enum value inside
its declaration. This is the proposed syntax:

\begin{alltt}\footnotesize
enum class Fruit \{ banana = 1, apple = 2, orange = 4, clementine = 8 \};
static\_assert(std::enum\_value\_v<Fruit::orange> == 4);
static\_assert(std::enum\_position\_v<Fruit::orange> == 2);
\end{alltt}

\noindent
This, together with \texttt{enum\_cardinality}, enables us to create
constructions such as the following:

\begin{alltt}\footnotesize
template<std::enum\_type T>
class EnumPriorityMapper \{
public:
  unsigned int GetPriority(T t) \{
    return mPriorities[std::enum\_position\_v<t>];
  \}
  void SetPriority(T t, unsigned int p) \{
    mPriorities[std::enum\_position\_v<t>] = p;
  \}
private:
  unsigned int mPriorities[std::enum\_cardinality\_v<T>];
\};
\end{alltt}

\noindent
Since both \texttt{enum\_cardinality} and \texttt{enum\_position} are defined
as compile-time constants, this code should never throw or index out-of-bounds.



\newpage
\section{Overview}

This is the complete list of proposed functions, types, and meta-functions:

\begin{alltt}\footnotesize
// header "<experimental/scoped\_enum>" -- or -- "<experimental/enum>"

// Alternatively:
export module std.scoped\_enum; // -- or --
export module std.enum;

namespace std \{

  // Evaluates whether T is the name of a declared scoped enum type
  template<typename T>
  concept enum\_type; // = /* implementation uncertain */

  // Checks if "V" is the underlying type for a declared scoped enum "T".
  template<enum\_type T, is\_integral\_v V>
  concept enum\_has\_type = std::is\_same\_v<std::underlying\_type\_t<T>, V>;

  // A forward-iterator of all values inside a declared scoped enum
  template<enum\_type T>
  struct enum\_iterator;

  // A container with all values inside a declared scoped enum,
  // with methods `begin()`, `end()`, `cbegin()`, and `cend()` for
  // supporting forward iterating through the values by using the
  // "enum\_iterator" defined above.
  template<enum\_type T>
  struct enum\_values;

  // Retrieves the underlying value represented by an enum value,
  // similar to "static\_cast<underlying\_type\_t<T>".
  template<enum\_type T, is\_integral\_v V>
  V enum\_value(T t);

  template<enum\_type T>
  bool enum\_try\_cast(integral auto v); // = /* implementation uncertain */

  // Retrieve the cardinality (ie. number of values) inside a declared
  // scoped enum.
  template<enum\_type T>
  struct std::enum\_cardinality \{ constexpr size\_t value; \};

  template<enum\_type T>
  size\_t std::enum\_cardinality\_v;

  // Retrieve a fully-qualified string representation of the provided
  // scoped enum value.
  template<enum\_type T>
  const char* enum\_string(T t); // = /* implementation compiler-specific */

  // Compare a scoped enum value with its underlying type (spaceship-operator)
  template<enum\_type T, is\_integral\_v V>
  weak\_ordering operator<=>(T lhs, V v);

  // Stream overload
  template<enum\_type T>
  ostream& operator<<(ostream& stream, T t) \{
    return stream << enum\_string<T>(t);
  \}

\} // namespace std

// Typesafe cast from underlying value to declared scoped enum value,
// including a check for whether the provided value is valid.
template<std::enum\_type T>
std::optional<T> enum\_cast(std::integral auto v) \{
  T out;
  if (std::enum\_try\_cast<T>(&out, v)) return out;
  else return std::nullopt;
\}

\end{alltt}




\end{document}
